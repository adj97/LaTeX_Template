\chapter{Introduction}
\label{ch:intro}
\graphicspath{{image_directory/introduction/}}

Traffic flow modelling is not just about understanding traffic dynamics, but traffic movement has significant links to economy, politics and the environment. Traffic and the mass movement of people and goods through cities and countries is a difficult problem to describe mathematically. The following however, indicates the development of research into traffic modelling, and why this field of study is important.

\section{History}
\label{sec:history}

	The first known scientific research into traffic flow theory was from Bruce D. Greenshields, who presented a publication \cite{Greenshields35} detailing his use of revolutionary photographic measurements and conclusions into flow relationships. Greenshields proposed a linear speed-density relationship and the fundamental $f=\rho\cdot u$ traffic relation \cite{Kuhne11}. Following World War II, the use of automobiles and traffic infrastructure had seen a significant development. The first international symposium on the theory of traffic flow modelling was held in December 1959 at the General Motors research laboratories in Warren, Michigan. The triennial symposia had to separate due to the expansion of the field of study and now many specialist conferences are held covering various elements of traffic flow modelling \cite{Dhingra11}. See papers from Kuhne \cite{Kuhne11} and Dhingra and Ishtiyaq \cite{Dhingra11}, from the 75$^{th}$ anniversary Greenshields' symposium for more detail into the origin, history and development of traffic flow modelling.  


\section{Motivation}
\label{sec:motivation}

	Traffic as a concept is widely important and hence requires research in order to understand certain processes and control or predict traffic situations. Accurate and reliable traffic flow modelling tools could benefit the environment by reducing unnecessary journey emissions, improve driver and pedestrian safety with well designed traffic intersections and integration with other infrastructure such as train networks and airways. With a recent increase in companies developing smart tools for road vehicles, an appropriate TFM software could assist autonomous driving decisions and GPS satellite navigation routing algorithms. With a better understanding of realistic traffic behaviour, city councils and infrastructure projects can be developed with more hindsight and improve road lifetimes helping to cut maintenance costs and road tax. Some of these applications are far away, however the tasks performed in this research are sophisticated enough to be able to run scenario tests. Such tests would be useful to prepare for special events that depend on organised traffic, such as the 200,000 visitors arriving at Glastonbury festival from all over the UK, and the contingency plans from Kent county council around the Port of Dover in the event of Brexit\footnote{See the term \emph{Dover Tap} for the UK government traffic assessment project (TAP) for port-bound vehicles on the A20 approaching Dover, available at \href{https://www.gov.uk/government/publications/dover-traffic-assessment-project-tap}{gov.uk}.}.
	\\ \\
	It is important to study hyperbolic PDE systems as compressible flow equations are formulated in this way \cite{Toro09}. Historically, compressible flows are of importance due to the engineering advances in high speed aerodynamics and aerospace \cite{Tu18}. Compressible flow analysis, hyperbolic PDEs in particular, can be used to solve traffic flow problems where cars along a single lane road can experience shocks and other phenomena from compressible fluid dynamics \cite{Thompson72}. Many problems arise such as oscillatory solutions due to shock discontinuities.

\section{Objectives and Structure}
\label{sec:objectives}

	Following this introduction to the origin and importance of modelling traffic flows, Chapter \ref{ch:lit} gives a review of many recent aspects of modern TFM, including a classification of modelling approaches with examples and previous studies into network flow modelling. The numerical approach is given in Chapter \ref{ch:approach}, where appropriate and or selected aspects of compressible fluid dynamics are explained, along with details of the integration of a macroscopic hyperbolic PDE solver and a probabilistic network flow model. These techniques are built into a Python modelling tool, of which the development is described in Chapter \ref{ch:development}. Here some guidance on preparing and using the tool are given along with a flow chart for steps of the solution process. This tool is implemented on certain problems and results of such are presented and discussed in Chapter \ref{ch:randd}. Studies of both theoretical and real road networks are completed and analysed for the influence of utilised numerical techniques on solutions, as well as computational time costs. The study concludes with Chapter \ref{ch:final}, where the main results, and suggestions for future development of this tool are given. Following this are the cited references, codes used for research\footnote{One should clone the \href{https://github.com/adj97/TFM_Thesis}{GitHub repository} for future development.} are listed for reference in Appendix \ref{ap:code}, reconstruction processes detailed in Appendix \ref{ap:reco}, and any other supporting material in Appendix \ref{ap:supp}.